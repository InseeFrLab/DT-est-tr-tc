\begin{tikzpicture}
%%%%%%%%%%%%%%%%%%%%%%%%
%%%%% formule gen %%%%%
%%%%%%%%%%%%%%%%%%%%%%%%

\node at (3,0) [generalbox, max width=8cm] (formulegen){
\[
\begin{cases}
I(\boldsymbol\theta,q,y_t,u_t)=\E{(\Delta^{q}(M_{\boldsymbol\theta} y_t-u_t))^{2}} \\
J(\boldsymbol\theta, f, \omega_1,\omega_2)=\int_{\omega_1}^{\omega_2} f\left[\phi_{\boldsymbol\theta}(\omega), \varphi_{\boldsymbol\theta} (\omega)\right] \ud \omega
\end{cases}
\]
\begin{align*}
\hat\theta\in\argmin &\sum\alpha_i I(\boldsymbol\theta,q_i,y_t,u_t^{(i)})+\beta_iJ(\boldsymbol\theta, f_i,\omega_{1}^{(i)}, \omega_{2}^{(i)})\\
&sc.\quad \boldsymbol C\boldsymbol \theta=\boldsymbol a&
\end{align*}%
};
\node[generaltitle] at (formulegen.north) (formulegentitle) {Formule générale};

%%%%%%%%%%%%%%%%%%%%%%%%
%%%%% Gray Thompson%%%%%
%%%%%%%%%%%%%%%%%%%%%%%%
\node (graythomsonA) at (-4,  9.7) {};
\node (graythomsonB) at (4.5,  5.3) {};
\node (graythomsonequiv) at (-3.7, 5.3) {};
\draw[multimodelbox] (graythomsonA) rectangle (graythomsonB);
\node[boxtitle2, below right] at (graythomsonA) (graythompsontitle) {Gray, Thompson (1996)};
\node at (0,9) [box, below, max width=6cm] (graythompson){
    Minimisation des révisions au filtre symétrique sous contrainte
};
%\node[boxtitle] at (graythompson.north) (graythompsontitle) {Gray, Thompson (1996)};


\node at (-2.5,7) [submodelbox, max width=3cm, below] (graythompson1){
    Filtre asymétrique sans biais 
    
$\implies \boldsymbol X'_{d}\hat{\boldsymbol \theta}=\boldsymbol e_1$
};
\node at (2,7) [submodelbox, max width=3cm, below] (graythompson2){
Filtre asymétrique de biais constant 

$\implies \boldsymbol X'_{d-1}\hat{\boldsymbol \theta}=\boldsymbol e_1$
};

\draw [edgesubmod] (graythompson) -- (graythompson1) node[right, midway]{Contrainte};
\draw [edgesubmod] (graythompson) -- (graythompson2);

%%%%%%%%%%%%%%%%%%%%%%%%%%
%%%%% Guggemos et al %%%%%
%%%%%%%%%%%%%%%%%%%%%%%%%%


\node at (0,-10.8) [box, max width=8cm, above] (guggemos){
$F_g(\boldsymbol\theta)=\sum_j\theta_j^2$

$S_g(\boldsymbol\theta)=\sum_j(\Delta^q\theta_j)^2$

$T_g(\boldsymbol\theta)=\int_{0}^{\omega_1}\phi_{\boldsymbol\theta}^2(\omega) \sin^2(\varphi_a(\omega)) \ud \omega$
\begin{align*}
\hat{\boldsymbol \theta}&\in\argmin \nu_1 F_g(\boldsymbol\theta) + \nu_2 S_g(\boldsymbol\theta)+ (1-\nu_1-\nu_2)T_g(\boldsymbol\theta)\\
&s.c.\quad \boldsymbol X'_d\boldsymbol \theta=\boldsymbol e_1
\end{align*}
};
\node[boxtitle] at (guggemos.north) (guggemostitle) {Guggemos \emph{et al} (2018)};


%%%%%%%%%%%%%%%%%%%%%%%%%%%%%
%%%%% Proietti et Luati %%%%%
%%%%%%%%%%%%%%%%%%%%%%%%%%%%%


\node (proiettiluatiA) at (7.3,-10.8) {};
\node (proiettiluatiB) at (24.8,-3.5) {};
\draw[multimodelbox] (proiettiluatiA) rectangle (proiettiluatiB);
\node[boxtitle2, below left] at (proiettiluatiB) (proiettiluatititle) {Proietti et Luati (2008)};

\node at (20,-9) [box, max width=8cm, inner ysep=2pt] (proiettiluati){
\begin{center}\centering \textbf{Modèle général}\end{center}\vspace{-0.2cm}
\[
\boldsymbol y=\boldsymbol U\boldsymbol \gamma+\boldsymbol Z\boldsymbol \delta+\boldsymbol \varepsilon,\quad \varepsilon\sim\mathcal N (0,\boldsymbol D)
\text{ et } \begin{bmatrix} \boldsymbol U & \boldsymbol Z \end{bmatrix} \subset \boldsymbol X
\]

Minimisation des révisions à $\boldsymbol\theta^s$ sous contrainte :
\[
\boldsymbol U'_p\boldsymbol \theta=\boldsymbol U'\boldsymbol\theta^s, \quad \boldsymbol U=\begin{bmatrix} \boldsymbol U_p \\ \boldsymbol U_f \end{bmatrix} \text{ avec }\boldsymbol U_p\text{ de }h+1+q\text{ lignes}
\]
};



\node at (24.5,-2) [box, max width=6cm, inner ysep=2pt, left] (genproiettiluati){
\begin{center}\centering \textbf{Extension en intégrant la timeliness}\end{center}\vspace{-0.2cm}

Minimisation  sous contrainte des révisions + $\alpha_T T_g(\boldsymbol\theta)$
};
%\node[boxtitle] at (proiettiluati.north) (proiettiluatititle) {Proietti et Luati (2008)};


%%%% DAF %%%%
\node at (11.3,-10) [submodelbox, max width=4.8cm, above] (daf){
\begin{center}\centering \textbf{DAF}\end{center}\vspace{-0.4cm}
 \[
  y_t=\gamma_0+\gamma_1 t+\gamma_2t^2+\gamma_3 t^3 +\varepsilon_t
  \]
  $\varepsilon_t$ bruit blanc et $\boldsymbol \theta$ préserve tendances cubiques
};


%%%% CQ %%%%
\node at (11.3,-8.4) [submodelbox, max width=4.8cm, above] (cq){
  \begin{center}\centering \textbf{CQ}\end{center}\vspace{-0.4cm}
  \[
  y_t=\gamma_0+\gamma_1 t+\gamma_2t^2+\delta t^3 +\varepsilon_t
  \]
  $\varepsilon_t$ bruit blanc et $\boldsymbol \theta$ préserve tendances quadratiques et dépend de $|\delta/\sigma|$
};

%%%% QL %%%%
\node at (11.3,-6.2) [submodelbox, max width=4.8cm, above] (ql){
  \begin{center}\centering \textbf{QL}\end{center}\vspace{-0.4cm}
  \[
  y_t=\gamma_0+\gamma_1 t+\delta t^2 +\varepsilon_t,
  \]
  $\varepsilon_t$ bruit blanc et $\boldsymbol \theta$ préserve tendances linéaires et dépend de $|\delta/\sigma|$
};


%%%% LC %%%%
\node at (20,-6) [submodelbox, max width=4.8cm, above] (lc){
\begin{center}\centering \textbf{LC/Musgrave}\end{center}\vspace{-0.4cm}
\[
y_t=\gamma_0+\delta t +\varepsilon_t, 
\]
$\varepsilon_t$ bruit blanc et $\boldsymbol \theta$ préserve constantes et dépend de $|\delta/\sigma|$
};

%%%% Liens
\draw [edgesubmod, max width = 2cm] (proiettiluati) -- (daf) node[below, midway] {
\[
\begin{cases}
  \boldsymbol D=\boldsymbol K^{-1}\\ 
  \boldsymbol U=\boldsymbol X_3\\ 
  \boldsymbol Z=\boldsymbol 0
\end{cases}
\]
};

\draw[edgesubmod, max width = 2cm] (proiettiluati) -- (cq) node[above right=-0.2cm,near end] {
\[
\begin{cases}
\boldsymbol U=\boldsymbol X_2\\ 
\boldsymbol Z=\boldsymbol x_3\\
\boldsymbol D=\sigma^2\boldsymbol I
\end{cases}
\]
};

\draw [edgesubmod, max width = 2cm] (proiettiluati.160) |- (ql) node[above, near end] {
\[
\begin{cases}
\boldsymbol U=\boldsymbol X_1\\ 
\boldsymbol Z= \boldsymbol x_2\\
\boldsymbol D=\sigma^2\boldsymbol I
\end{cases}
\]
};
\draw [edgesubmod, max width = 2cm] (proiettiluati) -- (lc) node[right, midway] {
\[
\begin{cases}
\boldsymbol U=\boldsymbol X_0\\ 
  \boldsymbol Z=\boldsymbol x_1\\
  \boldsymbol D=\sigma^2\boldsymbol I
\end{cases}
\]
};

%%%%%%%%%%%%%%%%
%%%%% RKHS %%%%%
%%%%%%%%%%%%%%%%

\node at (20,9.4) [box, max width=8cm, below] (rkhs){
$f_0(t)$ noyau continu, $P_i$ polynômes orthonormaux de $\mathbb L^2(f_0)$ et $K_p(t)=\sum_{i=0}^{p-1}P_i(t)P_i(0)f_0(t)$.
$$
\hat\theta_i = \frac{K_p(i/b)}{\sum_{j=-h}^q K_p(j/b)}
$$
$b$ choisi optimalement pour minimiser : 
\begin{itemize}[noitemsep,topsep=0pt]
\item l'erreur quadratique moyenne de révision ($b_{q,\gamma}$)
\item  l '\emph{accuracy} $A_w$ ($b_{q,G}$)
\item  la \emph{smoothness}  $S_w$ ($b_{q,s}$) 
\item  la \emph{timeliness} $T_w$  ($b_{q,\varphi}$)
\end{itemize}
};

\node[boxtitle] at (rkhs.north) (rkhstitle) {Dagum et Bianconcini (2008) --- RKHS};






%%%%%%%%%%%%%%%%
%%%%% Wildi %%%%%
%%%%%%%%%%%%%%%%


\node at (10,7.6) [box, max width=8cm] (trilemma){
\begin{align*}
A_w(\boldsymbol\theta)&= 2\int_0^{\omega_1}\left(\rho_s(\omega)-\rho_{\boldsymbol\theta}(\omega)\right)^{2}h(\omega)\ud\omega\\
T_w(\boldsymbol\theta)&= 8\int_0^{\omega_1}\rho_s(\lambda)\rho_{\boldsymbol\theta}(\lambda)\sin^{2}\left(\frac{\varphi_{\boldsymbol\theta}(\omega)}{2}\right)h(\omega)\ud\omega\\
S_w(\boldsymbol\theta)&= 2\int_{\omega_1}^\pi\left(\rho_s(\omega)^{2}-\rho_{\boldsymbol\theta}(\omega)\right)^{2}h(\omega)\ud\omega%\\
%R_w&= 8\int_{\omega_1}^\pi\rho_s(\lambda)\rho_{\boldsymbol\theta}(\lambda)\sin^{2}\left(\frac{\varphi_{\boldsymbol\theta}(\omega)}{2}\right)h(\omega)\ud\omega
\end{align*}
\[
\min\nu_1A_w + \nu_2T_w+(1-\nu_1-\nu_2)S_w
\]
};
\node[boxtitle] at (trilemma.north) (trilemmatitle) {Wildi, McElroy (2019)};



%%%%%%%%%%%%%%%%%%
%%%%% Liens %%%%%%
%%%%%%%%%%%%%%%%%%

\draw[edge] (genproiettiluati.west)-|(proiettiluati.158) node[pos = 0.57, right]{
$\alpha_T=0$
};

\draw[edge] (graythompson1.west) --++(-0.1,0)--++(0,-8) --++(10,0)|-(daf.west) node[pos=0.7, below, max width = 5cm]{
$\sigma^2=0, \boldsymbol \Omega=\boldsymbol K^{-1},$

$ d=3$
};

\draw[edge] (graythompson2.east) --++(2,0)--++(0,-1) --++(2.3,0)|-(cq.west) node[pos=0.7, below, max width = 5cm]{
$d=3$
};

\draw[edge] (graythompson2.east) --++(2,0)--++(0,-1) --++(2.3,0)|-(ql.west) node[pos=0.7, below, max width = 5cm]{
$d=2$
};

\draw[edge] (graythompson2.east) --++(2,0) --++(0,-1) --++(2.3,0) node[below left]{$\sigma^2=0, \boldsymbol \Omega=\boldsymbol K^{-1}$}--++ (0, -8) --++ (9,0) --++ (0,-1)-|(lc.north) node[pos=0.7, left, max width = 5cm]{
$d=1$
};


%%%%%%%%%%%%%%%%%%%
%%%%%% formule gen

%% gray thomson
\draw[edgegeneral] (formulegen.west)-|(graythomsonequiv) node[pos = 0.8, right]{
$y_t=\sum_{j=0}^d\beta_jt^j+\xi_t+\varepsilon_t$, avec $\varepsilon_t$ bruit blanc $\V{\varepsilon_t}=\sigma^2$ et $\xi_t$ non corrélé à $\varepsilon_t$,  $\boldsymbol \Omega = \V{(\xi_{t-h},\dots,\xi_{t+h})}$
\begin{align*}
\min R(\boldsymbol\theta)& =\E{\left(M_{\boldsymbol\theta^s} y_t - M_{\boldsymbol\theta} y_t\right)^2}
\\&= I(\boldsymbol\theta,\,0,\,y_t,\,M_{\boldsymbol\theta^s} y_t) \\
sc.\quad&  \boldsymbol X'_{d}\boldsymbol \theta=\boldsymbol e_1\text{ ou } \boldsymbol X'_{d-1}\boldsymbol \theta=\boldsymbol e_1
\end{align*}
};

%% guggemos
\draw[edgegeneral, max width = 8cm] (formulegen.west) -|(graythomsonequiv |- 52,  52 |-guggemos.north) node[pos = 0.9, right]{
$F_g(\boldsymbol\theta)=I(\boldsymbol\theta,0,y_t,\E{L_{\boldsymbol\theta} y_t})$

$S_g(\boldsymbol\theta)=I(\boldsymbol\theta,q,y_t,\E{L_{\boldsymbol\theta} y_t})$

$T_g(\boldsymbol\theta)=J(\boldsymbol\theta,f\colon(\rho,\varphi)\mapsto\rho^2\sin(\varphi)^2,0, \,\omega_1)$

$y_t=\sum_{j=0}^d\beta_jt^j+\varepsilon_t, \quad \varepsilon_{t}\overset{i.i.d}{\sim}\mathcal{N}(0,\sigma^{2})$
};

\draw[edgegeneral, max width = 8cm] (formulegen.east)-|(trilemma.south) node[pos=0.7,right]{
$\begin{cases}
    f_1(\rho,\varphi, \omega)=2\left(\rho_s(\omega)-\rho\right)^{2}h(\omega) \\
    f_2(\rho,\varphi, \omega)=8\rho_s(\omega)\rho\sin^{2}\left(\frac{\varphi}{2}\right)h(\omega)
\end{cases}$
\begin{align*}
A_w(\boldsymbol\theta)&= J(\boldsymbol\theta,f_1,0,\omega_1)\\
T_w(\boldsymbol\theta)&= J(\boldsymbol\theta,f_2,0,\omega_1)\\
S_w(\boldsymbol\theta)&= J(\boldsymbol\theta,f_1,\omega_1,\pi)\\
R_w(\boldsymbol\theta)&= J(\boldsymbol\theta,f_2,\omega_1,\pi).
\end{align*}
};

\draw[edgegeneral, max width = 4.cm] (formulegen.east)-|(rkhs.south) node[near end, right]{
$\min R(\boldsymbol\theta)$,  $A_w(\boldsymbol\theta)$,  $S_w(\boldsymbol\theta)$,  ou $T_w(\boldsymbol\theta)$

s.c.  $\theta_i=K_p(i/b)$ et $\sum\theta_i=1$
};

\draw[edgegeneral, max width = 8cm] (formulegen.east)-|(genproiettiluati.north) node[near end, right]{
$\min R(\boldsymbol\theta)+\alpha_TT_g(\boldsymbol\theta)$

s.c.  linéaires
};

\end{tikzpicture}
