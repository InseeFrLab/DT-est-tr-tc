\thispagestyle{fancy}

\hypertarget{ruxe9sumuxe9}{%
\section*{Résumé}\label{ruxe9sumuxe9}}
\addcontentsline{toc}{section}{Résumé}

Cette étude s'intéresse à différentes méthodes de construction de
moyennes mobiles pour l'estimation en temps réel de la tendance-cycle et
la détection rapide des points de retournement. Nous proposons une
comparaison des principales méthodes, en s'appuyant sur une formulation
générique de construction de moyennes mobiles. Nous décrivons également
deux prolongements possibles aux filtres polynomiaux locaux : l'ajout
d'un critère permettant de contrôler le déphasage (délai dans la
détection des points de retournement) et une façon de paramétriser
localement ces filtres. La comparaison empirique des méthodes montre que
: les problèmes d'optimisation de filtres issus des espaces de Hilbert à
noyau reproduisant (RKHS) augmentent le déphasage et les révisions des
estimations de la tendance-cycle ; modéliser des tendances polynomiales
trop complexes introduit plus de révisions sans diminuer le déphasage~;
pour les filtres polynomiaux, une paramétrisation locale permet une
réduction des révisions et du délai de détection des points de
retournement.

Mots clés : séries temporelles, tendance-cycle, désaisonnalisation,
points de retournement.

\hypertarget{abstract}{%
\section*{Abstract}\label{abstract}}
\addcontentsline{toc}{section}{Abstract}

This paper focuses on different approaches to build moving averages for
real-time trend-cycle estimation and fast turning point detection. We
propose a comparison of the main methods, based on a general unifying
framework to derive linear filters. We also describe two possible
extensions to local polynomial filters: the addition of a timeliness
criterion to control the phase shift (delay in the detection of turning
points) and a way to locally parameterize these filters. The empirical
comparison of the methods shows that: the optimization problems of the
filters from the Reproducing Kernel Hilbert Space (RKHS) theory increase
the phase shift and the revisions of the trend-cycle estimates; modeling
polynomial trends that are too complex introduces more revisions without
decreasing the phase shift; for polynomial filters, a local
parameterization reduces the phase shift and the revisions.

Keywords: time series, trend-cycle, seasonal adjustment, turning points.

JEL Classification: E32, E37.

\newpage
